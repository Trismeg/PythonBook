\chapter{Git}
Git is a distributed revision control system with an emphasis on speed, data integrity, and support for distributed, non-linear workflows.  Git was initially designed and developed by Linus Torvalds for Linux kernel development in 2005, and has since become one of the most widely adopted version control systems for software development.
\begin{marginfigure}%
  \includegraphics[width=\linewidth]{download.jpg}
  \label{fig:marginfig}
\end{marginfigure}
Please note the below assume you are using a Terminal shell in Linux or OSX operating system. If you are using Windows you will use "dir" instead of "ls" to list files using Command Terminal. Also note the slashes are different for writing file paths. Linux and OSX use forward slash / while Windows uses back slash.

\vspace{1cm}

\section{Introduce yourself to Git}

\begin{marginfigure}[60pt]

This is the first step you must do when using git for the first time. Tag your commits with Name and Email.
\end{marginfigure}
\begin{shaded}
\begin{verbatim}
$ git config --global user.name "Dr Doeg"
$ git config --global user.email doeg@example.com
\end{verbatim}
\end{shaded}


\newpage

\section{Basic Terminal Commands }

Using the terminal you may navigate the file directory.  Make, delete, move and rename files and directories.

\begin{margintable}[50pt]\index{unixfu}
  \footnotesize%
  \begin{center}
    \begin{tabular}{ll}
      \toprule
     Unix Command & Action \\
      \midrule
     \bf{cd}  & change directory        \\
    \bf{ls}  & list files        \\
     \bf{cp}  & list files        \\
      \bf{mv}  & move files        \\
       \bf{rm}  & remove files        \\
        \bf{rmdir}  & remove directory        \\
         \bf{touch}  & create file        \\
          \bf{nano}  & edit file        \\
           \bf{mkdir}  & create directory        \\
      \bottomrule
    \end{tabular}
  \end{center}
  \caption{A list of Unix shell commands.}
  \label{tab:font-sizes}
\end{margintable}


\begin{shaded}
\begin{verbatim}
$ cd path/to/project/folder
$ ls
$ cp filename ~/Location/newname
$ mv filename ~/Location/newname
$ rm filename
$ rmdir directoryname
$ touch filename
$ mkdir directoryname
$ nano filename
\end{verbatim}
\end{shaded}

\section{Git Repository}
You can get a Git project using two main approaches. The first takes an existing project or directory and imports it into Git. The second clones an existing Git repository from another server.

\section{Setting up a local Repository}
\marginnote[30pt]{Using the command line navigate to the project folder and initialize a git repository.}
\begin{shaded}
\begin{verbatim}
$ git init
\end{verbatim}
\end{shaded}
\marginnote[40pt]{Add files in the folder to the stage.\\ \ \\
\noindent Or add all the files.  }
\begin{shaded}
\begin{verbatim}
$ git add file2.jpg
$ git add .
\end{verbatim}
\end{shaded}
\marginnote[40pt]{Commit the additions.  }
\begin{shaded}
\begin{verbatim}
git commit -m "comment on the file changes"
\end{verbatim}
\end{shaded}

\newpage

\section{Push Your Local Repository to GitHub}

\marginnote[25pt]{\normalsize{Setup the remote repository location on GitHub using your account.}}
\begin{shaded}
\begin{verbatim}
$ git remote add origin https://github.com/<USER>/<REPO>.git
\end{verbatim}
\end{shaded}

\marginnote[40pt]{If you already set up the remote and want to change it use "set-url".}
\begin{shaded}
\begin{verbatim}
$ git remote set-url origin https://..../<USER>/<REPO>.git
\end{verbatim}
\end{shaded}
\marginnote[40pt]{Push the committed structure to the remote server.}
\begin{shaded}
\begin{verbatim}
$ git push origin master
\end{verbatim}
\end{shaded}

\vspace{1cm}

\section{Cloning an Existing Repository From GitHub }


\marginnote[40pt]{Navigate to the desired location in file structure.}
\begin{shaded}
\begin{verbatim}
$ cd path/to/whereUwant/folder
\end{verbatim}
\end{shaded}
\marginnote[40pt]{Set the location on the GitHub server to place the repositiory.}
\begin{shaded}
\begin{verbatim}
$ git clone https://github.com/<USER>/<REPO>.git
\end{verbatim}
\end{shaded}

\vspace{1cm}

\section{Working With Branches }
Version control is one of the great powers of git.

\begin{margintable}[80pt]\index{branchfu}
  \footnotesize%
  \begin{center}
    \begin{tabular}{ll}
      \toprule
     Unix Command & Action \\
      \midrule
     \bf{branch}  & list branches       \\
    \bf{branch} <NAME>  & create new branch        \\
    \bf{checkout} <NAME>  & switch to new branch        \\
     \bf{merge} <NAME>  & merge branch with current      \\
      \bf{branch -m} <NAME>  & rename current branch \\
      \bf{branch -D} <NAME>  & delete branch       \\
      \bottomrule
    \end{tabular}
  \end{center}
  \caption{A list of git commands for version control.}
  \label{tab:font-sizes}
\end{margintable}

\begin{shaded}
\begin{verbatim}
$ git branch
$ git branch branchname
$ git checkout branchname
$ git merge branchname
$ git branch -m newbranchname
$ git branch -D branchname
\end{verbatim}
\end{shaded}

\vspace{1cm}

\section{Updating an Existing Repository From GitHub }

\marginnote[20pt]{The sophisticated way to update uses fetch, reviews changes and merges those onto the master branch.  The alt the current project folder from the GitHub remote server.}
\begin{shaded}
\begin{verbatim}
$ git fetch
$ git pull -u origin master
\end{verbatim}
\end{shaded}

\vspace{1cm}

\section{Get Git, Github and More on Git}

\marginnote[10pt]{Download git and register an account at GitHub. Look at the official documentation for more information.}
https://git-scm.com/downloads  \\
\noindent https://github.com/  \\
\noindent  https://git-scm.com/book/en/v2

\bibliography{sample-handout}
\bibliographystyle{plainnat}



\end{document}






