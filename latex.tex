	\begin{marginfigure}%
		\includegraphics[width=\linewidth]{lat.png}
		\caption{This is the logo of latex.}
		\label{fig:marginfig}
	\end{marginfigure}
	
	
	\normalsize
	
	%this generates 1cm of vertical space
	\vspace{1cm}
	\section{what is latex}
	
	LaTeX is a document preparation system for high-quality typesetting. It is most often used for medium-to-large technical or scientific documents but it can be used for almost any form of publishing.(Basically its a high-tech version of a pdf document maker that can do much more stuff than normal word documents.)

	

	
	\vspace{1cm}
	
	\section{this is how you start a latex file.}
	
	\marginnote[40pt]{Here you start you file by telling what kind of file you are making and the title and author of it. And begin the document.}
	\marginnote[40pt]{ And begin the document. remenber to have enddocument at the end of the paper.}

	\begin{framed}
		\begin{verbatim}
		\documentclass{tufte-handout}
		
		\title{Latex}
		
		\author[The Academy]{Tony/Zekang Lin}
		
		\begin{document}
		\end{verbatim}
	\end{framed}
		\section{this is some useful things you can do in latex file.}
		
		\marginnote[40pt]{you can use margin figure or figure to include pictures as a note or just something that you want to show.it will auto matically lable the picture as figure + the number of the image.}
		\marginnote[40pt]{use marginnote to add notes.}
		\marginnote[40pt]{shaded can help you shade what you are going to write and verbatim will allow you to write your codes in latex.}
		
		\begin{framed}
			\begin{verbatim}
\begin{marginfigure}%
\includegraphics[width=\linewidth]{XXXXX.png}
\caption{This is the logo of latex.}
\label{fig:marginfig}
\end{marginfigure}


\marginnote[30pt]{DXXXXXXXXXXXXXXXXXXXXXXXXXXXXX}


\begin{shaded}
\begin{verbatim}
			\end{verbatim}
		\end{framed}
		
\vspace{1cm}
\section{why do we use latex.}
	\begin{verbatim}
Latex is easy to use and there are many stuff that latex will 
automaticallydo for you, such as it will automatically write 
the date that you last edited and autometically lable the number 
of images you added to the paper.
	\end{verbatim}
	
	\vspace{1cm}
	\section{this is what a Latex paper can look like:}
	\marginnote[40pt]{}
		\begin{figure}
			\includegraphics[width=\linewidth]{latresult.png}
			\caption{this is what latex paper can look like.}
			\label{fig:marginfig}
		\end{figure}
	
